\documentclass[12pt]{article}
\usepackage{amsmath}
\usepackage{amssymb}
\usepackage{amsthm}
\usepackage{mathrsfs}
\usepackage{hyperref}
\usepackage{graphicx}
\usepackage{geometry}
\geometry{margin=1in}

% Define theorem styles
\theoremstyle{definition}
\newtheorem{definition}{Definition}
\newtheorem{remark}{Remark}
\theoremstyle{plain}
\newtheorem{theorem}{Theorem}
\newtheorem{lemma}[theorem]{Lemma}

\title{Rupture Dynamics and Memory Persistence in Discontinuous Systems}
\author{}
\date{}

\begin{document}
\maketitle

\begin{abstract}
We develop a rigorous mathematical framework for analyzing systems exhibiting discontinuous transitions while maintaining information persistence. By introducing the notion of memory trace subspaces and resonance mappings, we establish conditions under which global coherence emerges from local discontinuities. We prove the existence of persistent subspaces and demonstrate applications to quantum mechanics, neural networks, and dynamical systems.
\end{abstract}

\section{Introduction}

Discontinuous transitions in physical and mathematical systems often present challenges for traditional analytical methods. This work introduces a formal framework for studying systems where information persists across such discontinuities, which we term "rupture points." Our approach unifies treatments across multiple domains, from quantum measurement to neural computation.

\subsection{Motivation}

Consider a dynamical system that undergoes discrete transitions. Classical theory typically struggles with such discontinuities, yet many natural systems exhibit robust information preservation across similar transitions. Examples include:
\begin{itemize}
    \item Quantum systems under measurement
    \item Neural networks during learning events
    \item Phase transitions in physical systems
    \item Cognitive state transitions in biological systems
\end{itemize}

Our framework provides a unified mathematical treatment of such phenomena.

\section{Mathematical Framework}

\subsection{Preliminary Definitions}

\begin{definition}[Strong Continuity with Uniform Boundedness]
Let \( X, Y \) be Banach spaces. A family of operators \( \{T(t)\}_{t \in \mathbb{R}} \) from \( X \) to \( Y \) is said to be strongly continuous and uniformly bounded if:
\begin{enumerate}
    \item For each \( x \in X \), the map \( t \mapsto T(t)x \) is continuous in the norm topology of \( Y \).
    \item There exists a constant \( C > 0 \) such that \( \|T(t)\| \leq C \) for all \( t \in \mathbb{R} \).
\end{enumerate}
\end{definition}

\begin{definition}[Sectional Limits in the Strong Operator Topology]
For an operator-valued function \( M(t): \mathbb{R} \to \mathcal{L}(V) \), where \( \mathcal{L}(V) \) denotes the space of bounded linear operators on \( V \), the sectional limits at \( t_0 \) are defined as:
\[
M(t_0^-) = \lim_{t \to t_0^-} M(t), \quad M(t_0^+) = \lim_{t \to t_0^+} M(t),
\]
where the limits are taken in the strong operator topology. That is, for all \( x \in V \):
\[
\lim_{t \to t_0^\pm} \|M(t)x - M(t_0^\pm)x\| = 0.
\]
\end{definition}

\subsection{Rupture Systems}

\begin{definition}[Rupture System]
A rupture system is a tuple \( (V, M, \mathcal{T}, S, \mathcal{F}) \), where:
\begin{itemize}
    \item \( V \) is a Banach space.
    \item \( \mathcal{T} = \{t_1,\ldots,t_k\} \subset \mathbb{R} \) is a finite set of rupture points.
    \item \( M: \mathbb{R} \setminus \mathcal{T} \to \mathcal{L}(V) \) is a strongly continuous operator-valued function, where \( \mathcal{L}(V) \) is the space of bounded linear operators on \( V \).
    \item \( S \subseteq V \) is a closed subspace (the memory trace subspace) such that \( S \) is invariant under \( M(t) \) for all \( t \notin \mathcal{T} \).
    \item \( \mathcal{F} = \{f_{ij}\}_{i<j} \) is a collection of resonance functions.
\end{itemize}
The following axioms must be satisfied:
\end{definition}

\subsection{Axioms}

\paragraph{Axiom 1 (Local Invertibility with Uniform Boundedness):}
For each \( t \notin \mathcal{T} \), there exists \( \varepsilon > 0 \) and a bounded operator-valued function \( N(s): (t-\varepsilon, t+\varepsilon) \to \mathcal{L}(V) \) such that:
\begin{enumerate}
    \item \( M(s)N(s) = N(s)M(s) = I_V \) for all \( s \in (t-\varepsilon, t+\varepsilon) \).
    \item The map \( s \mapsto N(s) \) is strongly continuous.
    \item \( \|N(s)\| \leq C \), where \( C > 0 \) is uniform across \( (t-\varepsilon, t+\varepsilon) \).
    \item At the boundaries \( t-\varepsilon \) and \( t+\varepsilon \), the operator \( N(s) \) remains well-defined and satisfies the above properties.
\end{enumerate}

\paragraph{Axiom 2 (Memory Persistence with Continuity of Projections):}
There exists a projection \( P_S: V \to S \) satisfying:
\begin{enumerate}
    \item \( P_S \) is bounded with \( \|P_S\| = 1 \).
    \item \( P_S \) is strongly continuous: for any \( x \in V \), the map \( x \mapsto P_S(x) \) is continuous in the norm topology.
    \item For all \( v \in S \) and \( t_i \in \mathcal{T} \), both sectional limits \( M(t_i^-)v \) and \( M(t_i^+)v \) exist in the strong operator topology.
    \item \( P_S(M(t_i^-)v) = P_S(M(t_i^+)v) \) for all \( v \in S \).
\end{enumerate}

\paragraph{Axiom 3 (Resonance):}
For each pair \( t_i < t_j \in \mathcal{T} \), the resonance function \( f_{ij}: V \to V \) satisfies:
\begin{enumerate}
    \item \( f_{ij} \) is bounded and strongly continuous.
    \item \( f_{ij}(M(t_i^+)v) = M(t_j^-)v \) for all \( v \in S \).
    \item \( f_{ij} \circ f_{jk} = f_{ik} \) for all \( t_i < t_j < t_k \).
\end{enumerate}

\section{Applications}

\subsection{Quantum Mechanics}
In quantum measurement theory, rupture points correspond to state collapse events:
\[
M(t) = U(t)PU^\dagger(t).
\]

\subsection{Neural Networks}
Neural networks exhibit rupture points during learning events.

\subsection{Dynamical Systems}
For a dynamical system \( (X, \phi_t) \), rupture systems characterize bifurcation points.

\section{Conclusions and Future Directions}

This framework opens several promising directions:
\begin{enumerate}
    \item Classification of rupture systems by persistent subspace structure.
    \item Quantitative measures of information preservation.
    \item Applications to machine learning architecture design.
    \item Connections to category theory via functor properties.
\end{enumerate}

\end{document}