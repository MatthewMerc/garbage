\documentclass[12pt]{article}
\usepackage{amsmath}
\usepackage{amsthm}
\usepackage{amssymb}
\usepackage{mathrsfs}
\usepackage{graphicx}
\usepackage{hyperref}
\usepackage{geometry}
\usepackage{tikz}
\usepackage{xcolor}
\usetikzlibrary{arrows,shapes,positioning,decorations.pathreplacing,backgrounds,fit,shapes.geometric}

% Define colors for visualizations
\definecolor{rupturered}{RGB}{255,89,94}
\definecolor{memorygreen}{RGB}{138,201,38}
\definecolor{flowblue}{RGB}{25,130,196}
\definecolor{resonancegold}{RGB}{255,202,58}

% Define theorem environments
\theoremstyle{definition}
\newtheorem{definition}{Definition}
\newtheorem{remark}{Remark}
\theoremstyle{plain}
\newtheorem{theorem}{Theorem}
\newtheorem{lemma}[theorem]{Lemma}
\newtheorem{axiom}{Axiom}

\title{A Unified Framework for Rupture Dynamics and Memory Persistence in Discontinuous Systems}
\author{Building on work by Terry Tao}
\date{\today}

\begin{document}

\maketitle

\begin{abstract}
We present a comprehensive mathematical framework for analyzing systems that exhibit discontinuous transitions while maintaining information persistence. By introducing and formalizing the concepts of memory trace subspaces and resonance mappings, we establish rigorous conditions under which global coherence emerges from local discontinuities. Our framework unifies treatments across multiple domains, from quantum measurement to neural computation, providing a robust mathematical foundation for understanding information preservation across rupture events.
\end{abstract}

\section{Introduction}

Many natural and artificial systems exhibit behavior characterized by sudden transitions or discontinuities, yet maintain certain invariant properties across these ruptures. Traditional analytical methods often struggle to capture this duality of discontinuous change and information persistence. Our work provides a unified mathematical treatment of such phenomena.

\subsection{Motivation}

Consider a dynamical system undergoing discrete transitions where certain information persists despite discontinuous change. Examples include:
\begin{itemize}
    \item Quantum systems during measurement events
    \item Neural networks during learning phase transitions
    \item Physical systems at critical points
    \item Cognitive state transitions in biological systems
\end{itemize}

\section{Mathematical Preliminaries}

\begin{definition}[Strong Continuity with Uniform Boundedness]
Let $(X, \|\cdot\|_X)$ and $(Y, \|\cdot\|_Y)$ be Banach spaces. A family of operators $\{T(t)\}_{t \in \mathbb{R}}$ from $X$ to $Y$ is strongly continuous and uniformly bounded if:
\begin{enumerate}
    \item For each $x \in X$, the map $t \mapsto T(t)x$ is continuous in $Y$'s norm topology
    \item $\exists C > 0$ such that $\|T(t)\| \leq C$ for all $t \in \mathbb{R}$
\end{enumerate}
\end{definition}

\section{Rupture Systems}

\begin{definition}[Rupture System]
A rupture system is a tuple $(V, M, \mathcal{T}, S, \mathcal{F})$ where:
\begin{itemize}
    \item $V$ is a Banach space
    \item $\mathcal{T} = \{t_1,\ldots,t_k\} \subset \mathbb{R}$ is a finite set of rupture points
    \item $M: \mathbb{R} \setminus \mathcal{T} \to \mathcal{L}(V)$ is strongly continuous
    \item $S \subseteq V$ is the memory trace subspace
    \item $\mathcal{F} = \{f_{ij}\}_{i<j}$ is a collection of resonance functions
\end{itemize}
\end{definition}

\section{Core Axioms}

\begin{axiom}[Local Invertibility]
For each $t \notin \mathcal{T}$, $\exists \varepsilon > 0$ and $N(s): (t-\varepsilon, t+\varepsilon) \to \mathcal{L}(V)$ such that:
\begin{enumerate}
    \item $M(s)N(s) = N(s)M(s) = I_V$ for $s \in (t-\varepsilon, t+\varepsilon)$
    \item $s \mapsto N(s)$ is strongly continuous
    \item $\|N(s)\| \leq C$ uniformly
\end{enumerate}
\end{axiom}

\begin{axiom}[Memory Persistence]
There exists a projection $P_S: V \to S$ satisfying:
\begin{enumerate}
    \item $\|P_S\| = 1$
    \item $P_S$ is strongly continuous
    \item Sectional limits $M(t_i^\pm)v$ exist for $v \in S$
    \item $P_S(M(t_i^-)v) = P_S(M(t_i^+)v)$ for $v \in S$
\end{enumerate}
\end{axiom}

\begin{axiom}[Resonance]
For $t_i < t_j \in \mathcal{T}$, $f_{ij}: V \to V$ satisfies:
\begin{enumerate}
    \item $f_{ij}$ is bounded and strongly continuous
    \item $f_{ij}(M(t_i^+)v) = M(t_j^-)v$ for $v \in S$
    \item $f_{ij} \circ f_{jk} = f_{ik}$ for $t_i < t_j < t_k$
\end{enumerate}
\end{axiom}

\begin{figure}[h]
\centering
\begin{tikzpicture}[scale=1.2]
    % Time axis
    \draw[->] (-4,0) -- (4,0) node[right] {$t$};
    \draw[->] (0,-2) -- (0,2) node[above] {$M(t)$};
    
    % Flow before first rupture
    \draw[thick, flowblue] (-4,1) .. controls (-3,1.2) and (-2.5,0.8) .. (-2,1);
    
    % First rupture point
    \draw[thick, rupturered] (-2,-1.5) -- (-2,1.5);
    \filldraw[rupturered] (-2,0.2) circle (2pt);
    
    % Flow between ruptures
    \draw[thick, flowblue] (-2,-0.5) .. controls (-1,-0.7) and (0,-0.3) .. (1,-0.5);
    
    % Second rupture point
    \draw[thick, rupturered] (1,-1.5) -- (1,1.5);
    \filldraw[rupturered] (1,0.2) circle (2pt);
    
    % Flow after second rupture
    \draw[thick, flowblue] (1,0.5) .. controls (2,0.3) and (3,0.7) .. (4,0.5);
    
    % Memory trace
    \draw[dashed, memorygreen] (-4,0.7) -- (4,0.7);
    
    % Resonance connections
    \draw[->, resonancegold, thick] (-2,1) to[bend left=30] (1,0.5);
    \draw[->, resonancegold, thick] (1,-0.5) to[bend left=30] (2,0.5);
    
    \node[below] at (-2,0) {$t_1$};
    \node[below] at (1,0) {$t_2$};
    \node[right] at (4,0.7) {$S$};
\end{tikzpicture}
\caption{Visualization of a rupture system showing flow dynamics (blue), rupture points (red), memory trace (green), and resonance connections (gold).}
\end{figure}

\section{Main Results}

\begin{theorem}[Persistent Subspace]
For any rupture system satisfying the axioms, there exists a non-trivial subspace $W \subset V$ such that:
\begin{enumerate}
    \item $W \subseteq S$
    \item Information encoded in $W$ persists across all rupture points
    \item $\dim(W) > 0$
\end{enumerate}
\end{theorem}

\begin{proof}
Consider the intersection of kernels:
\[ W = \bigcap_{i=1}^k \ker(P_S \circ M(t_i^+) - P_S \circ M(t_i^-)) \cap S \]
The result follows from the finite dimensionality of $V$ and the continuity properties of $P_S$ and $M$.
\end{proof}

\section{Applications}

\subsection{Quantum Mechanics}
In quantum measurement theory, rupture points correspond to measurement events. The memory trace subspace captures quantum numbers that persist through measurement, while resonance functions describe the relationship between pre- and post-measurement states:
\[ M(t) = U(t)PU^\dagger(t) \]
where $U(t)$ is the unitary evolution and $P$ is the measurement projection.

\subsection{Neural Networks}
In neural computation, rupture points correspond to discrete learning events, while the memory trace subspace captures preserved features and learned patterns. Resonance functions describe how learned information transforms across architectural changes.

\section{Future Directions}

This framework opens several promising research directions:
\begin{enumerate}
    \item Classification of rupture systems by persistent subspace structure
    \item Development of quantitative measures for information preservation
    \item Application to machine learning architecture design
    \item Extension to infinite-dimensional cases
    \item Connection to category theory via functor properties
\end{enumerate}

\end{document}