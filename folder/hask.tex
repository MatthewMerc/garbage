\documentclass{article}
\usepackage{amsmath}
\usepackage{amsthm}
\usepackage{listings}
\usepackage{xcolor}

\lstset{
    language=Haskell,
    basicstyle=\ttfamily\small,
    keywordstyle=\color{blue},
    stringstyle=\color{red},
    commentstyle=\color{green!60!black},
    numberstyle=\tiny\color{gray},
    numbers=left,
    breaklines=true,
    breakatwhitespace=true,
    showspaces=false,
    showtabs=false,
    frame=single
}

\title{Simplified Surreal Numbers Implementation}
\author{}
\date{}

\begin{document}
\maketitle

\section{Data Structure}
The simplified surreal number structure is defined as an algebraic data type:

\begin{equation}
\text{Surreal} = \begin{cases}
\text{Zero} \\
\text{One} \\
\text{NegOne} \\
\text{Succ}(x) & \text{where } x \text{ is Surreal} \\
\text{Pred}(x) & \text{where } x \text{ is Surreal} \\
\text{Sum}(x, y) & \text{where } x, y \text{ are Surreal} \\
\text{Prod}(x, y) & \text{where } x, y \text{ are Surreal}
\end{cases}
\end{equation}

\section{Basic Operations}

\subsection{Ordering}
For surreal numbers $x$ and $y$, the ordering is defined as:

\begin{align*}
\text{compare}(\text{Zero}, \text{Zero}) &= \text{EQ} \\
\text{compare}(\text{One}, \text{One}) &= \text{EQ} \\
\text{compare}(\text{NegOne}, \text{NegOne}) &= \text{EQ} \\
\text{compare}(\text{Succ}(x), \text{Succ}(y)) &= \text{compare}(x, y) \\
\text{compare}(\text{Pred}(x), \text{Pred}(y)) &= \text{compare}(x, y) \\
\text{compare}(\text{Zero}, \text{One}) &= \text{LT} \\
\text{compare}(\text{One}, \text{Zero}) &= \text{GT} \\
\text{compare}(\text{Zero}, \text{NegOne}) &= \text{GT} \\
\text{compare}(\text{NegOne}, \text{Zero}) &= \text{LT}
\end{align*}

\subsection{Normalization}
The normalization function reduces surreal numbers to their simplest form:

\begin{align*}
\text{normalize}(\text{Zero}) &= \text{Zero} \\
\text{normalize}(\text{One}) &= \text{One} \\
\text{normalize}(\text{NegOne}) &= \text{NegOne} \\
\text{normalize}(\text{Succ}(\text{Zero})) &= \text{One} \\
\text{normalize}(\text{Pred}(\text{Zero})) &= \text{NegOne} \\
\text{normalize}(\text{Sum}(x, y)) &= \text{addNormalized}(\text{normalize}(x), \text{normalize}(y)) \\
\text{normalize}(\text{Prod}(x, y)) &= \text{multNormalized}(\text{normalize}(x), \text{normalize}(y))
\end{align*}

\section{Arithmetic Operations}

\subsection{Addition}
For normalized surreal numbers $x$ and $y$:
\begin{equation}
\text{addNormalized}(x, y) = \text{fromInt}(\text{evalToInt}(x) + \text{evalToInt}(y))
\end{equation}

\subsection{Multiplication}
For normalized surreal numbers $x$ and $y$:
\begin{equation}
\text{multNormalized}(x, y) = \text{fromInt}(\text{evalToInt}(x) \times \text{evalToInt}(y))
\end{equation}

\subsection{Negation}
\begin{align*}
\text{negate}(\text{Zero}) &= \text{Zero} \\
\text{negate}(\text{One}) &= \text{NegOne} \\
\text{negate}(\text{NegOne}) &= \text{One} \\
\text{negate}(\text{Succ}(x)) &= \text{Pred}(\text{negate}(x)) \\
\text{negate}(\text{Pred}(x)) &= \text{Succ}(\text{negate}(x)) \\
\text{negate}(\text{Sum}(x, y)) &= \text{Sum}(\text{negate}(x), \text{negate}(y)) \\
\text{negate}(\text{Prod}(x, y)) &= \text{Prod}(\text{negate}(x), y)
\end{align*}

\section{Integer Conversion}
The conversion between integers and surreal numbers is defined as:

\begin{equation}
\text{fromInt}(n) = \begin{cases}
\text{Zero} & \text{if } n = 0 \\
\text{One} & \text{if } n = 1 \\
\text{NegOne} & \text{if } n = -1 \\
\text{Succ}(\text{fromInt}(n-1)) & \text{if } n > 0 \\
\text{Pred}(\text{fromInt}(n+1)) & \text{if } n < 0
\end{cases}
\end{equation}

\section{Successor Function}
The nth successor of a surreal number $x$ is defined as:
\begin{equation}
\text{succ}'(n, x) = \underbrace{\text{Succ}(\text{Succ}(\cdots \text{Succ}(x)\cdots))}_{n \text{ times}}
\end{equation}

\end{document}