\documentclass[12pt]{article}
\usepackage{amsmath, amssymb, amsthm}
\usepackage{geometry}
\usepackage{graphicx}
\usepackage{hyperref}
\usepackage{tikz}
\geometry{margin=1in}

% Custom commands
\newcommand{\ord}[0]{\mathcal{O}}
\newcommand{\sur}[0]{\mathcal{S}}
\newcommand{\house}[0]{\mathcal{H}}

% Theorem Environments
\newtheorem{definition}{Definition}[section]
\newtheorem{example}{Example}[section]
\newtheorem{remark}{Remark}[section]
\newtheorem{theorem}{Theorem}[section]
\newtheorem{lemma}{Lemma}[section]

\title{Time-Evolving Paradoxical Spaces:\\A Dynamic Graph Framework with Surreal Distances}
\author{Author Name}
\date{\today}

\begin{document}

\maketitle

\begin{abstract}
We present a novel mathematical framework for modeling time-evolving graphs with surreal ordinal distances and recursive paradoxical spaces. Building on surreal number theory, dynamic graph systems, and memory-like structures from sheaf theory, we analyze the evolution of spaces over time. This approach incorporates infinitesimal and transfinite changes, resonance phenomena, and singularities. Applications include non-Euclidean geometries, dynamic systems, and semiotic paradoxes inspired by recursive spaces such as the \emph{House of Leaves}.
\end{abstract}

\tableofcontents

\section{Introduction}

Spaces that evolve recursively over time, exhibit paradoxical growth, or involve transfinite measures require new mathematical tools. Traditional graph theory is insufficient for representing:
\begin{itemize}
    \item Flows through spaces with infinitesimal and infinite distances,
    \item Dynamic transformations of nodes and edges over time,
    \item Singularities and resonance points where spatial continuity breaks.
\end{itemize}

Our framework builds on three primary tools:
\begin{enumerate}
    \item \textbf{Surreal Numbers:} Recursive structures that model infinite and infinitesimal changes,
    \item \textbf{Dynamic Graphs:} Graphs with evolving nodes, edges, and surreal weights,
    \item \textbf{Monodromy and Memory:} Structures inspired by sheaf theory to capture time evolution and resonance.
\end{enumerate}

This paper introduces the mathematical definitions, properties, and examples for analyzing such spaces.

\section{Surreal Numbers and Ordinals}

\subsection{Surreal Numbers}
\begin{definition}[Surreal Numbers]
A surreal number $\sur$ is defined recursively as:
\[
\sur = \{ L \mid R \}, \quad \text{where } L, R \subseteq \sur \text{ and } L < R.
\]
\end{definition}

Surreal numbers extend real numbers to include infinitesimal ($\epsilon$) and transfinite ($\omega$) values.

\begin{example}[Basic Surreal Numbers]
\[
0 = \{ \mid \}, \quad 1 = \{ 0 \mid \}, \quad -1 = \{ \mid 0 \}, \quad \frac{1}{2} = \{ 0 \mid 1 \}.
\]
Infinitesimals and infinities are given as:
\[
\epsilon = \{ 0 \mid \}, \quad \omega = \{ \mid 0 \}.
\]
\end{example}

\subsection{Ordinal Arithmetic}
Surreal numbers naturally generalize ordinals, allowing operations such as addition, multiplication, and exponentiation.

\begin{definition}[Ordinal]
An ordinal $\ord$ is expressed as:
\[
\ord = \omega^w + k, \quad w, k \in \mathbb{N}.
\]
\end{definition}

\begin{remark}
Ordinal arithmetic includes:
\[
\omega + 1 > \omega, \quad \omega \cdot 2 = \omega, \quad \omega^\omega > \omega^2.
\]
\end{remark}

\section{Dynamic Graphs with Surreal Distances}

\subsection{Graph Framework}
We define a dynamic graph $\house = (V, E, W)$:
\begin{itemize}
    \item $V$: Nodes (e.g., spaces),
    \item $E$: Edges connecting nodes,
    \item $W$: Weights as surreal distances.
\end{itemize}

\begin{definition}[Surreal Distance]
The weight $w_{uv} \in W$ of an edge $(u, v)$ is a surreal ordinal:
\[
w_{uv} = \omega^w + k \text{ or an infinitesimal } \epsilon.
\]
\end{definition}

\subsection{Edge Types and Recursions}
Edges can take the following forms:
\begin{enumerate}
    \item \textbf{Single Edge:} A standard edge with a surreal weight,
    \item \textbf{Infinite Edge:} A recursive edge repeating transfinite times,
    \item \textbf{Fractional Edge:} A fractional path with rational weight,
    \item \textbf{Paradoxical Edge:} An edge with undefined behavior.
\end{enumerate}

\begin{example}[The Five-Minute Hallway]
This hallway grows infinitely when entered:
\[
\text{Nodes: Entrance } \to \text{ Hallway } \to \text{ Abyss}.
\]
Edges are weighted as:
\[
\text{Entrance} \xrightarrow{\omega} \text{Hallway}, \quad \text{Hallway} \xrightarrow{\omega^2} \text{Abyss}.
\]
\end{example}

\section{Time Evolution and Monodromy}

\subsection{Transformation Rules}
A time-evolving graph $\house(t)$ evolves based on transformations:
\[
\text{Transformation: } \quad T : E \to E', \quad w' = w + \epsilon.
\]
Transformations may include:
\begin{itemize}
    \item Local perturbations,
    \item Discontinuous jumps,
    \item Resonance interactions.
\end{itemize}

\begin{definition}[Monodromy]
The \emph{monodromy} of a time-evolving graph captures how nodes and edges transform over time, including:
\[
\text{Singularities: Special times where continuity breaks}.
\]
\end{definition}

\subsection{Conditions for Stability}
\begin{theorem}[Stability Condition]
A dynamic graph $\house(t)$ is stable if:
\[
\forall u, v \in V: \quad \text{flow through } E \text{ remains bounded as } t \to \infty.
\]
\end{theorem}

\section{Paradoxical Spaces and Sheaf Memory}

\begin{example}[Mirror Room]
The mirror room generates infinite reflections at infinitesimal steps:
\[
\text{Origin} \xrightarrow{\epsilon} \text{First Reflection} \xrightarrow{\epsilon} \text{Second Reflection} \to \dots.
\]
The sequence collapses to an abyss as $n \to \infty$.
\end{example}

\begin{remark}
The \emph{sheaf-like memory} structure ensures that information about nodes persists through singularities and transformations.
\end{remark}

\section{Conclusion}

We have developed a framework for time-evolving graphs with surreal distances, transformations, and paradoxical structures. Applications span recursive geometries, dynamic systems, and semiotic spaces.

Future directions include:
\begin{itemize}
    \item Extending monodromy to higher-dimensional spaces,
    \item Applying the framework to physical systems with non-standard flows,
    \item Investigating connections to game theory and logic.
\end{itemize}

\bibliographystyle{plain}
\begin{thebibliography}{9}
\bibitem{conway} Conway, J. H. (1976). \textit{On Numbers and Games}.
\bibitem{graph} Harary, F. (1969). \textit{Graph Theory}.
\bibitem{house} Danielewski, M. Z. (2000). \textit{House of Leaves}.
\end{thebibliography}

\end{document}