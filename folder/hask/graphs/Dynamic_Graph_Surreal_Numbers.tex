\documentclass[12pt]{article}
\usepackage{amsmath, amssymb, amsthm}
\usepackage{geometry}
\usepackage{graphicx}
\usepackage{hyperref}
\geometry{margin=1in}

\title{\textbf{Recursive Games, Surreal Payoffs, and Impossible Spaces:\\
A Unified Framework for Dynamic Graph Systems}}
\author{Your Name}
\date{\today}

\begin{document}

\maketitle

\begin{abstract}
We propose a novel mathematical framework that unifies recursive game theory, dynamic graph systems, and surreal-valued payoffs. By integrating infinitesimal and transfinite numbers into game dynamics, we model decision-making across evolving graphs where perturbations, singularities, and resonance points govern equilibria. Monodromy, inspired by algebraic topology, describes the behavior of games under recursion, while impossible spaces provide a conceptual metaphor for decision systems that collapse or stabilize across infinite paths. This work bridges game theory, graph dynamics, and recursive systems, with applications in AI training, climate policy, and strategic negotiations.
\end{abstract}

\section{Introduction}
Classical game theory models static equilibria, but many real-world systems involve recursion, evolution, and infinite decision layers. Negotiations, climate agreements, and AI training systems exhibit delays, adjustments, and perturbations that can only be captured by a recursive framework.

We introduce a new framework for recursive games defined on dynamic graphs:
\begin{itemize}
    \item Nodes represent decision states that evolve under recursive strategies.
    \item Edges carry surreal-valued weights, including infinitesimal penalties (\(\epsilon\)) and transfinite rewards (\(\omega\)).
    \item Perturbations dynamically adjust nodes, edges, and payoffs, leading to emergent equilibria at resonance points.
\end{itemize}

Inspired by *impossible spaces* like those in *House of Leaves* \cite{danielewski2000house}, we model recursive decision-making in systems where strategies collapse, stabilize, or oscillate under infinite recursion. Monodromy maps track strategy evolution across singularities, providing tools to analyze game equilibria under perturbations.

\section{Recursive Games on Dynamic Graphs}
Let \( G_t = (N_t, E_t) \) be a graph at time \( t \), where:
\begin{itemize}
    \item \( N_t \) is the set of nodes (decision states),
    \item \( E_t \) is the set of edges with weights \( w_t \in \mathbb{S} \) (surreal numbers).
\end{itemize}

\subsection{Surreal-Valued Payoffs}
Surreal numbers generalize payoffs to include:
\begin{itemize}
    \item **Infinitesimal penalties**: \( \epsilon \), where \( \epsilon > 0 \) but \( \epsilon < r \, \forall r > 0 \),
    \item **Transfinite rewards**: \( \omega \), representing infinite magnitudes,
    \item **Collapsed costs**: Negative infinities (\(-\omega\)) when recursion breaks down.
\end{itemize}

The payoff for a recursive strategy is:
\[
u(t) = \sum_{i=0}^t w_i - k\epsilon, \quad \text{where } k \to \infty \text{ collapses the strategy.}
\]

\subsection{Game Evolution Rules}
The graph \( G_t \) evolves according to:
\[
G_{t+1} = f(G_t, \Delta_t),
\]
where \( \Delta_t \) introduces perturbations:
\begin{itemize}
    \item Adding new nodes: Expanding the strategy space,
    \item Adjusting edge weights: Rebalancing payoffs under recursion,
    \item Removing edges: Eliminating dominated strategies.
\end{itemize}

\subsection{Example: Recursive Negotiation}
Consider two agents negotiating recursively over time:
\begin{itemize}
    \item Cooperation yields payoff \( 1 \) at time \( t \),
    \item Delay imposes an infinitesimal penalty \( \epsilon \),
    \item Infinite delays collapse the negotiation into \( -\omega \).
\end{itemize}
The recursive payoff is:
\[
u(t) = \begin{cases}
1 - k\epsilon & \text{for finite delays,} \\
-\omega & \text{for infinite delays.}
\end{cases}
\]

Equilibrium occurs at **resonance points** where:
\[
\text{Cost(Cooperate)} = \text{Cost(Delay)}.
\]

\section{Monodromy and Resonance Analysis}
Monodromy tracks changes in graph states under recursion:
\[
\Phi: G_0 \to G_t,
\]
where \( \Phi \) describes how nodes, edges, and weights transform as recursion progresses.

\subsection{Resonance Points}
Resonance points stabilize recursive dynamics. For a node \( n \), resonance occurs when:
\[
\sum_{t=0}^\infty \Delta u(t) = 0,
\]
where \( \Delta u(t) \) captures recursive changes in payoffs.

\subsection{Singularities}
Singularities arise when strategies collapse:
\[
u(t) \to -\omega.
\]
These points correspond to boundaries where recursion destabilizes the system.

\section{Impossible Spaces: Conceptual Metaphors}
Inspired by *House of Leaves* \cite{danielewski2000house}, we interpret recursive graphs as impossible spaces:
\begin{itemize}
    \item Nodes represent decision "rooms" that evolve recursively,
    \item Edges describe surreal-valued paths that connect states,
    \item Singularities correspond to spatial boundaries dissolving under infinite recursion.
\end{itemize}

Impossible spaces model phenomena such as:
\begin{itemize}
    \item Infinite delays collapsing strategies (\(-\omega\)),
    \item Emerging pathways under perturbations,
    \item Recursive oscillations stabilizing at resonance points.
\end{itemize}

\section{Applications}
\subsection{AI Training Systems}
Recursive games provide tools for AI agents balancing:
\begin{itemize}
    \item Immediate exploration (finite payoffs),
    \item Long-term exploitation (transfinite gains),
    \item Infinitesimal penalties for delays.
\end{itemize}

\subsection{Strategic Negotiations}
Recursive delays in negotiations impose increasing costs:
\[
u(t) = 1 - k\epsilon, \quad \text{where infinite delays collapse to } -\omega.
\]

\subsection{Climate Policy Models}
Infinite recursion models procrastination in climate agreements. Delays increase costs until catastrophic collapse:
\[
u(t) \to -\omega \quad \text{as } t \to \infty.
\]

\section{Future Work}
This framework opens avenues for:
\begin{itemize}
    \item Stochastic extensions for recursive games,
    \item Empirical validation in multi-agent AI systems,
    \item Topological analysis of dynamic graphs under perturbations.
\end{itemize}

\section{Conclusion}
We developed a unified framework for recursive games, surreal-valued payoffs, and dynamic graph systems. Monodromy and resonance describe equilibria under infinite recursion, while impossible spaces serve as conceptual tools for understanding boundary-breaking dynamics. Applications include AI, negotiations, and climate policy, with future work focusing on stochastic and empirical extensions.

\begin{thebibliography}{9}
\bibitem{conway1976surreal} J. H. Conway, \textit{On Numbers and Games}, Academic Press, 1976.
\bibitem{danielewski2000house} M. Z. Danielewski, \textit{House of Leaves}, Pantheon Books, 2000.
\bibitem{von2007game} J. von Neumann and O. Morgenstern, \textit{Theory of Games and Economic Behavior}, Princeton University Press, 2007.
\end{thebibliography}

\end{document}